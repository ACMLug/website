\documentclass[a4paper]{article}
\usepackage{linuxdoc-sgml}
\usepackage{qwertz}
\usepackage{url}
\usepackage[latin1]{inputenc}
\usepackage{t1enc}
\usepackage{babel}
\usepackage{epsfig}
\usepackage{null}
\def\addbibtoc{
\addcontentsline{toc}{section}{\numberline{\mbox{}}\relax\bibname}
}%end-preamble
\title{uiuc.sw.linux FAQ}
\author{\onlynameurl{The UIUC Linux Users' Group}}
\date{v0.2 Sunday, 4 April 1999}
\abstract{A collection of all the Frequently Asked Questions that pop up on the
newsgroup uiuc.sw.linux.}


\begin{document}
\maketitle
\tableofcontents

\section{UIUC Linux Basics}


\subsection{What is Linux?}

Linux is a free UNIX(tm)-like operating system developed by a loosely-knit
team of programmers all around the world.  Originally created in 1991 by 
Linus Torvalds, it contains all the features you would expect from a
modern fully-fledged UNIX, including preemptive multitasking,
multi-threading, symmetric multiprocessing, virtual memory, shared libraries,
demand paging, shared copy-on-write executables, advanced memory management,
and one of the fastest TCP/IP network stacks available.

It is distributed under the GNU General Public License.




\subsection{Is Linux UNIX?}

Technically, no.  UNIX is a registered trademark of the X/Open Company
Ltd., and Linux cannot be called UNIX without permission from them.
Furthermore, Linux is not based off of any of the source code from the 
original versions of UNIX, unlike the free BSD variants. However, you
will probably find that many, if not most, of the standard UNIX
applications are available under Linux; quite often porting software to
Linux simply requires little more than a recompile.




\subsection{Why should I use Linux?}

Linux is a technically advanced, Open Source(tm), UNIX-like operating
system.  It is stable, efficient, and runs on many different
architectures. Many pieces of software, including the Linux kernel
itself, can be obtained free of  charge over the Internet.  It is
especially popular as a server, providing services such as DNS, NIS,
HTTP, NFS, and SMB (Windows(tm) file and printer sharing), and as a
high-power low-cost workstation.




\subsection{What is a distribution?}

Linux, in general, usually refers to the kernel of the operating
system.  Many other components are necessary for the use of a computer,
such as applications, libraries, configuration files, startup files,
etc.  Distributions, such as
\onlynameurl{Red Hat},
\onlynameurl{Debian},
SuSE,
\onlynameurl{Slackware},
\onlynameurl{Stampede}, 
and Caldera, package all these components together with the Linux kernel
and provide mechanisms for its installation.  Each distribution has its
own "personality" and is tailored towards a certain group of people.

A more complete list of distributions is available at \onlynameurl{http://www.linuxhq.com/dist-index.html}.




\subsection{What distribution do you recommend?}

There is no one "best" distribution.  For beginners, we often recommend
Red Hat or SuSE; for slightly more advanced users, we also recommend
Debian.  Try as many different distributions out as you can; soon you
will find which appeals to you the most.




\subsection{How do I install Linux?}

After deciding on a distribution, you can usually install Linux using a
number of different methods, including CD-ROM, floppy, FTP, NFS, and
another partition on your hard drive.  This method is highly dependent
on the distribution, and we recommend that you visit your chosen
distribution's web site and follow their instructions.




\subsection{What about FreeBSD, OpenBSD, and NetBSD?}

These are other excellent, free UNIX-like operating systems available
for many different architectures.  They have slightly different feels
than Linux, but many run Linux binaries.  They also have a less
restrictive license, allowing redistribution without releasing the
source code.  We encourage you to try them out as well.

Please visit
\onlynameurl{http://www.freebsd.org},
\onlynameurl{http://www.openbsd.org},
or \onlynameurl{http://www.netbsd.org}.




\subsection{What is LUG?}

LUG refers to the UIUC Linux Users' Group.  We usually meet at 8pm on
Tuesdays in 1102 DCL.  Feel free to drop in with questions,
comments, or simply to talk about what's new with Linux; new members
are always welcome.  Our web page is \onlynameurl{http://www.acm.uiuc.edu/lug/}.




\subsection{What is Encap?}

From the \onlynameurl{UIUC Encap Home Page}:

"Encaps" are precompiled, easily installable packages in a format that
makes software maintenance very easy. Encaps of many public domain
utilities are available for most major operating systems.




\subsection{What is X?}

X, otherwise known as the X Window System or X11, is a network
transparent window system which runs on a wide range of computing
and graphics machines.  It is also architecture-independent, so you can
remotely run an application from a Sun to your Linux box (for example).
The version of X that usually comes with Linux is called \onlynameurl{XFree86}.




\section{Installation and Booting}


\subsection{How can I dual boot Windows 9x and Linux?}

Please see the Linux+Win95 HOWTO at \onlynameurl{http://metalab.unc.edu/LDP/HOWTO/mini/Linux+Win95.html}.




\subsection{How can I dual boot Windows NT(tm) and Linux?}

If you plan on using LILO as your primary bootloader, the instructions
should be nearly exactly the  same as in "How do I dual boot Windows 9x
and Linux?"  If you plan on using the NT bootloader, please see the
Linux+NT-Loader HOWTO at \onlynameurl{http://metalab.unc.edu/LDP/HOWTO/mini/Linux+NT-Loader.html}.




\section{Command Line and Administration}


\subsection{How can I list all the files in the current directory except for . and ..?}

If you are using GNU ls, then ls -A will do it.  Otherwise, use the
regular expression "* .{[}\^{}.]* ..?*".




\subsection{How do I set environment variables under sh derivatives (sh, bash, zsh, etc)?}

To set environment variables in the Bourne-derived shells, use the
export command, as in:

export VARIABLE=value

Place this in the proper dot-file for your shell (.profile, .bashrc,
.zshrc, .zshenv, etc).




\subsection{How do I set environment variables under csh derivatives (csh, tcsh)?}

The C-Shell uses a different syntax than the Bourne shells, and so you
will need to use setenv, as in:

setenv VARIABLE value

Again, place this in the proper dot-file for your shell (.chsrc,
.login, etc)




\subsection{How can I set my time from an atomic clock?}

From a posting to uiuc.sw.linux by Mark Roth,

First obtain the package xntp.  Then run "ntpdate ntp.uiuc.edu" once
on bootup, and run xntpd to keep the clock synced.  Put "server
ntp.uiuc.edu" in /etc/ntp.conf so xntpd knows where to get its time
information.




\subsection{Why should I use ssh instead of telnet or rsh?}

Traditionally, telnet and rsh do not encrypt their connections.
Therefore, anybody who can access a computer between yours and the
destination could easily grab your password if you use those utilities.
Ssh protects against this attack, along with IP spoofing, IP source
routing, DNS spoofing, and X attacks based on X authentication data.
Furthermore, ssh automatically sets up X forwarding for you; there is no
need to mess with xhost or xauth.

To learn more about ssh, visit \onlynameurl{the ssh FAQ}.




\subsection{How do I upgrade my kernel?}

Please see the Kernel HOWTO at \onlynameurl{http://metalab.unc.edu/LDP/HOWTO/Kernel-HOWTO.html}.




\section{The X Window System}


\subsection{How do I configure X?}

If you are running Red Hat, you can try the command Xconfigurator.
Otherwise, try using the program XF86Setup (a graphical configuration
program that uses TCL/TK).  Since XF86Setup does not exist everywhere,
xf86config, a command-line tool, can also be used to configure your X
setup.

Alternatively, you can edit your XF86Config (which is usually in /etc/
or /etc/X11/) with your favorite text editor.




\subsection{How do I upgrade X?}

If you are using a distribution such as Red Hat or Debian, we recommend
that you look for the appropriate packages and upgrade using that
method.  Specifically, the Red Hat packages can probably be found at
\onlynameurl{ftp://updates.redhat.com}.  If you want to compile the source
code yourself, please look for the packages at \onlynameurl{ftp://ftp.xfree86.org} or a mirror.




\subsection{What is a window manager?}

A window manager is the program that determines how things like window
borders, icons, and menus are drawn.  By changing the window manager you 
can make your desktop look like nearly anything you want.  For example,
there are window managers that emulate the look and feel of Windows 95,
the Macintosh(tm), NeXT(tm), etc.




\subsection{Which window manager do you recommend?}

\onlynameurl{Window Maker} is a very 
popular window manager that has a NeXT-like interface.  Other
popular window managers include \onlynameurl{Enlightenment}, FVWM, IceWM, and AfterStep.  Like distributions,
no one window manager is the "best" for everyone.  Information about
many different window managers, with screenshots, is available at
\onlynameurl{http://www.plig.org/\~{}xwinman/}.




\subsection{How do I run X at a different color depth?}

The easiest way to do this is to start X by using "startx -- -bpp 16"
where 16 is the color depth you wish to run.  Alternatively, you can
manually edit the startx script to always use this color depth, or
manipulate your XF86Config (which is usually in /etc/ or /etc/X11/) by
removing all references to other color depths.  For example, the
relevant section in my XF86Config looks like:

\begin{tscreen}
\begin{verbatim}
Section "Screen"
    Driver      "accel"
    Device      "Matrox Millennium II 4MB"
    Monitor     "Gateway2000 Vivitron 17"
    Subsection "Display"
        Depth       16
        Modes       "1152x864" "1024x768" "800x600" "640x480" "640x400"
    EndSubsection
EndSection
\end{verbatim}
\end{tscreen}


Be sure to back up your XF86Config before attempting modifications.




\subsection{How do I remotely run X applications?}

If you use ssh, everything should already be set up.  Otherwise, a quick 
tutorial on xauth will tell you something like:

\begin{tscreen}
\begin{verbatim}
localmachine > xauth list | grep localmachine
localmachine:0 MIT-MAGIC-COOKIE-1 102384102398410923841023481234
... other stuff ...
localmachine > telnet remotemachine
remotemachine > xauth add localmachine:0 MIT-MAGIC-COOKIE-1 102384102398410923841023481234
remotemachine > export DISPLAY=localmachine:0
\end{verbatim}
\end{tscreen}


That's all there is to it.  Of course, if you are using csh or tcsh, you
will have to use setenv instead of export.

More information is available at \onlynameurl{http://www.acm.uiuc.edu/workshops/security/x.html}.




\subsection{XDM is ugly.  What can I replace it with?}

Some popular alternatives include
\onlynameurl{kdm},
\onlynameurl{gdm},
\onlynameurl{login.app},
and \onlynameurl{wdm}.




\subsection{What are X resources?}

X maintains a global, network-aware resource database that is often
referred to as the X resource database.  Items are added to this
database using the xrdb command, usually from files with names like
Xdefaults and Xresources. X resources are used by most X programs for
their configuration, and allow extreme levels of customization.
However, learning how to manipulate them is somewhat intimidating.

An example of a small .Xresources file:

\begin{tscreen}
\begin{verbatim}
rxvt.background:   Black
rxvt.foreground:   AntiqueWhite
rxvt.cursorColor:  Red3
rxvt.font:         fixed
\end{verbatim}
\end{tscreen}


Look into the commands xrdb and editres to find out how to manipulate X
resources.




\subsection{Why should I use xauth instead of xhost?}

If you use xhost to remotely display X applications, then you are
leaving yourself wide open to easy X session hijacking, password
grabbing, and the like.  This is because the line "xhost +hostname"
allows anybody at the computer named "hostname" to access your X
session.  "xhost +" allows anybody in the world access to your X session 
-- a very bad thing indeed.

Xauth is much better than xhost because it requires the attacker to
exert more effort to compromise security.  By creating a passcode, and
using that passcode on the remote computer to access your X session, you 
are preventing people who don't know the passcode from getting at your
machine.  Of course, if the connection to the remote computer is
unencrypted, they could conceivably steal the passcode.  However, that's 
the least of your worries in that case.

More information is available at \onlynameurl{http://www.acm.uiuc.edu/workshops/security/x.html}.




\subsection{How do I get my IntelliMouse(tm) wheel to work in X?}

For a PS/2 mouse:

First, you'll need to edit your XF86Config file (in /etc or
/etc/X11) to set the Protocol in the Pointer section to IMPS/2 {[}1].
Now, go to \onlynameurl{http://solaris1.mysolution.com/\~{}jcatki/imwheel/} and download
the imwheel package.  Follow the installation instructions, and the
wheel should now work.  Note that the same procedure is used for
Logitech wheel mice, too.

{[}1] Here's the Pointer section of XF86Config after following the
instructions for installing imwheel.  

\begin{tscreen}
\begin{verbatim}
Section "Pointer"
   Protocol        "IMPS/2"
   Device          "/dev/psaux"
   BaudRate        1200
   ZAxisMapping    4 5
   Buttons         3
EndSection
\end{verbatim}
\end{tscreen}





\section{Desktop Environments}


\subsection{What are GNOME and KDE?}

GNOME and KDE aim to be fully-functional, integrated desktop
environments.  They consist of a number of different programs that all
look and act similarly, and can be customized in a common way.  GNOME
can be found at \onlynameurl{http://www.gnome.org/}, and KDE can be found at \onlynameurl{http://www.kde.org/}.




\subsection{Are GNOME and KDE window managers?}

No, they are not.  GNOME and KDE work with a number of different window
managers, and encompass much more than the simple manipulation of
windows.  However, each one of them recommends a window manager that
"works best."  For GNOME, the recommended window manager is \onlynameurl{Enlightenment}, and for KDE
it is the included KWM.  Other window managers, such as \onlynameurl{Window Maker}, integrate well
with both GNOME and KDE.




\subsection{Which of GNOME and KDE do you recommend?}

As of this writing, KDE seems to be further ahead in terms of end-user
ease of use and installation, while GNOME has numerous technical
merits.  Try both out and find which appeals to you more.




\section{Linux Applications}


\subsection{Why are my Netscape buttons black and white?}

This problem seems to be specific to certain color depths.  Try changing 
your color depth to a different value.




\subsection{Where can I find information about filtering mail using procmail?}

Visit Joe Gross's excellent procmail tutorial at \onlynameurl{http://www.stimpy.net/procmail/tutorial/}.




\subsection{Is ICQ available for linux?}

Yes!  There are numerous clones available (not to mention Mirabilis'
java version, which is not very good).  See \onlynameurl{http://www.portup.com/\~{}gyandl/icq/}.




\subsection{Is there a Netmeeting compatible program for Linux?}

Netmeeting uses H.323 for audio conferencing, and there is work to develop
a free implementation at \onlynameurl{Open H.323}. They have some source/binaries available, and people
have used them to successfully talk to NetMeeting 3.0. Unfortunately,
Microsoft used a proprietary system for the video.




\subsection{What word processors are available for Linux?}

Word Perfect, Star Office, ApplixWare, LyX, KOffice.




\subsection{What spreadsheets are available for Linux?}

Star Office, ApplixWare, KOffice, Gnumeric.




\subsection{Where can I look for other applications?}

A good starting point is either \onlynameurl{http://freshmeat.net/} or \onlynameurl{http://www.linuxapps.com/}.




\section{UIUC-Specific Information}


\subsection{How do I run Mentor Graphics remotely?}

The primary problem in getting Mentor Graphics to work remotely seems to 
be having the right fonts installed.  To get the right fonts, you must
run the mgc\_font\_collect script on a machine that has Mentor Graphics
installed, and then copy these fonts to your computer and add them to
your X font database.




\subsection{How can I use my Linux machine as my primary e-mail machine?}

Surprisingly this is pretty straight-forward. In Redhat, there is almost no
dreaded sendmail hackery to be done. First off, if you don't have a static IP,
you're going to be SOL, and you're going to have to settle for using
fetchmail or POP and changing your FROM address in your mail client. In which 
case, refer to those questions in this FAQ.

If you do have a static IP address, and have a registered domain name (note,
this is probably only possible if you live in private housing. In which case
you can usually request a static IP), then all you have to do is edit
/etc/sendmail.cw and on a line by itself, put in the domainname that points to
your static IP address, then restart sendmail (something along the lines of
/etc/rc.d/init.d/sendmail restart for System V style inits). 






\subsection{How can I change my from header in e-mails?}

Cool mail clients first: in Mutt, simply add 

my\_hdr From: someuser@somewhere.net (Some Name)

to .muttrc. In Pine, you can set the hostname of the from field but apparently
not much else. Go to (S)etup, then (C)onfig, then edit the second field down. 
In Netscape you may specify the whole email address under
Edit-{$>$}Prefrences-{$>$}Identity.




\section{Unanswered (so far) Questions}


\subsection{How can I set up fetchmail/POP to read mail on my own machine?}






\subsection{How do I connect to a PPTP server?}






\subsection{How can I set up a serial terminal into my linux box?}






\subsection{How do I connect to UIUC using PPP?}






\subsection{How do I get gradebook to work under wine?}






\subsection{How do I use the URH printers from Linux?}






\subsection{How can I make my mouse "left-handed"?}






\subsection{Where can I get more information? (Books, web sites, howtos, etc)}






\subsection{How can I replicate one hard drive's contents to another?}






\subsection{How can I specify what kind of mouse I want to use on the console and X?}






\subsection{What is RPM?  How do I use it?}






\subsection{Why do fonts suck? (truetype, 100dpi fonts, unscaled)}






\subsection{How do I get my backspace key to work in X, Netscape, and rxvt?}






\subsection{What are shadow passwords?}






\subsection{How do I install and configure shadow passwords?}






\subsection{Where can I get information about UIUC site-licensed packages such as Motif?}






\subsection{How can I access my Linux partition from Windows?}






\subsection{How can I use LPR from Windows?}






\subsection{Why does Netscape crash on any Java applet (old libc5 version conflict, may not be applicable anymore)}






\subsection{How can I debug LILO problems (LI, etc)}






\subsection{How do I change the background in X?}






\subsection{How do I have Linux boot up into a X login/password prompt?}






\subsection{How can I change my resolution in X?}






\subsection{How do I change my path?}






\subsection{How do I get NUM LOCK to default to on?}






\subsection{My (very large, greater than 8.4GB) hard drive isn't supported.}






\subsection{How do I change my Emacs colors?}






\subsection{How do I change the default colors of my XTerm or RXVT?}






\section{About this Document}


\subsection{Who helped with creating this document?}

This document was originally created by Steven Engelhardt (sengelha@uiuc.edu),
with help from Andrew Ho, Mark Roth, Todd Ostermeier, Jay Tamboli, David
Terrell, and others.

The current maintainer is Steven Engelhardt.




\subsection{How was this document created?}

This document was originally written in SGML.  Using the Debian \onlynameurl{sgmltools} package, it is
converted into the various formats for distribution, which includes
HTML, ASCII, and PostScript.




\section{Trademarks}

UNIX

Open Source

IntelliMouse

Windows

Macintosh

NeXT

NT



\end{document}
